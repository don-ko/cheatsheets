\section{Confidence Intervals}
Getting from Sample Dist. to Parameters: Estimations
\paragraph{Point Estimate}
Ideal Properties:
\begin{itemize}
	\item Unbiased (Centred at parameter)
	\item Small Standard Deviation ($\therefore$ Sample Mean over Median)
\end{itemize}
\paragraph{Confidence Interval}
\begin{itemize}
	\item Indicates precision
	\item Interval around the point estimate (Margin of Error)
	\item Associated with certain degree of confidence ($\approx0.95$)\\
		The probability that it contains $p$
\end{itemize}
If we generated a $(1-\alpha)$ interval using the same method over many random
samples, to estimate many population parameters, in the long run, $(1-\alpha)$ of
those intervals will contain the population parameter.
\paragraph{Confidence Interval: Proportions}
\textcolor{Bittersweet}{Assumptions}:
\begin{itemize}
	\item Data obtained by randomisation
	\item Distribution is $\rsimdots$ Normal ($np(1-p)\ge5$)\\
		For Binomial $\approx$ Normal, $s_{\hat{p}}\approx se$ approximation,
\end{itemize}
$(1-\alpha)$ Confidence Interval:
\begin{align*}
	\hat{p}\pm &\underbrace{q_{1-\frac{\alpha}{2}}(se)}\quad\textcolor{Bittersweet}{s_{\hat{p}}\approx se}=\sqrt{\frac{\hat{p}(1-\hat{p})}{n}}\\
	&\text{Margin of Error}
\end{align*}
\[\textcolor{Blue}{\textup{Margin of Error} \le W}\iff n\ge\frac{q_{1-\frac{\alpha}{2}}^2}{W^2}\hat{p}(1-\hat{p})\]
\begin{align*}
	n \Uparrow&\Rightarrow MoE\Downarrow\\
	(1-\alpha)\Uparrow&\Rightarrow MoE\Uparrow\\
	p\Uparrow&\Rightarrow \hat{p}\Uparrow\Rightarrow MoE\Uparrow
\end{align*}
$n$ ultimately depends on costs and limitations. Rectify:\\
If $p\approx0\vee p\approx1, +2 successes, +2 failures$
\paragraph{Confidence Interval: Mean}
\textcolor{Bittersweet}{Assumptions}:
\begin{itemize}
	\item Data obtained by randomisation
	\item Population Distribution $\rsimdots$ Normal\\
		Robust, but not to \textcolor{Bittersweet}{outliers}:
		\begin{itemize}
			\item Summary statistics $\overline{X}$ and $s_X$ sensitive to outliers.
			\item $\overline{X}$ no longer $\approx\mu_{\overline{X}}=\mu$
		\end{itemize}
\end{itemize}
\textcolor{OliveGreen}{Robustness}
Confidence Interval is robust wrt. the normality assumption
$\Rightarrow$ Performs adequately even when assumption is \textcolor{Bittersweet}{modestly} violated\\
$(1-\alpha)$ Confidence Interval:
\begin{align*}
	\overline{X}\pm&\underbrace{t_{df=n-1,1-\frac{\alpha}{2}}(se)}\quad s_{\overline{X}}\approx se=\frac{s_X}{\sqrt{n}}\\
	&\quad\text{Margin of Error}
\end{align*}
\textcolor{Gray}{$s_X$ is a point estimator of $\sigma$.}\\
For $df\ge30$ ($n>30$), and $\mu\pm3\sigma\approx Range(X)$
\[\textcolor{Blue}{\textup{Margin of Error} \le W}\iff n\ge\frac{\sigma^2q_{1-\frac{\alpha}{2}^2}}{W^2}\]
\begin{align*}
	n \Uparrow&\Rightarrow MoE\Downarrow\\
	(1-\alpha)\Uparrow&\Rightarrow MoE\Uparrow\\
	\sigma^2\Uparrow&\Rightarrow s^2\Uparrow\Rightarrow MoE\Uparrow
\end{align*}
\paragraph{t-Distribution}
Distribution to allow generalisation for small sample sizes but assumes normal distribution
\[t_{df}\qquad df\in\mathbb{R}\]
\[lim_{df\to\infty}t_{df}=Norm(0,1)\]
\[t_{df=30}\approx Norm(0,1)\]
\begin{enumerate}
	\item Bell-shaped
	\item Slightly thicker tails than normal
	\item Shows more variability than normal
\end{enumerate}